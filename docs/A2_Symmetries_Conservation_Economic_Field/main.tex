\documentclass[12pt,a4paper]{article}

\usepackage[utf8]{inputenc}
\usepackage[T1]{fontenc}
\usepackage{amsmath,amssymb,amsfonts}
\usepackage{bm}
\usepackage{geometry}
\usepackage{graphicx}
\usepackage{hyperref}

\geometry{margin=2.4cm}

\title{\textbf{Symmetries and Conservation Laws in the Economic Field:\\
Global Phase, Translations, and Coherent Invariants}}

\author{Ivan Salines --- Independent Researcher}
\date{November 2025}

\begin{document}
\maketitle

\begin{abstract}
This document develops the symmetry structure of the economic field and the 
associated conservation laws. Building on the density--phase description 
introduced in the foundational chapter, we identify three principal symmetries:
global phase shifts, spatial translations, and time translations. 
These symmetries are interpreted in economic terms as invariance under 
re-labellings of orientation, homogeneity of the underlying domain, and 
persistence of internal structure. 
Each symmetry leads to a conserved quantity: total density, a momentum-like 
flow invariant, and a Hamiltonian functional. 
Together, these results provide a geometric backbone for coherent economic 
dynamics, linking field symmetries to observable macroscopic invariants.
\end{abstract}

\tableofcontents

\section{Introduction}

The economic field is defined in terms of a density field $\rho(x,t)$ and a 
phase field $\theta(x,t)$, whose joint dynamics are governed by a 
Lagrangian functional. 
Beyond the specific form of the Lagrangian, its \emph{symmetries} encode 
deep structural properties of the system. 
They determine which quantities are conserved, how coherence can persist, and 
which transformations leave the underlying economics unchanged.

This chapter examines three principal symmetries:
\begin{itemize}
    \item global phase symmetry,
    \item spatial translation symmetry,
    \item time translation symmetry.
\end{itemize}
Each symmetry has a clear economic interpretation and an associated conserved 
quantity, in direct analogy with Noether's theorem in field theory.

\section{Lagrangian Structure of the Economic Field}

We recall the Lagrangian density for a multi-component economic field:
\begin{equation}
\mathcal{L}
= \sum_{i=1}^{N} \left[
 \rho_i\,\frac{\partial \theta_i}{\partial t}
 - \frac{1}{2}\,\rho_i\,|\nabla \theta_i|^2
 - V_i(\rho_i)
\right]
- \sum_{1 \le i < j \le N}
   J_{ij}\,\rho_i\,\rho_j\,
   \bigl[\,1 - \cos(\theta_i - \theta_j)\,\bigr],
\label{eq:LagrangianEconomic}
\end{equation}
where:
\begin{itemize}
    \item $\rho_i(x,t)$ is the density field of subsystem $i$,
    \item $\theta_i(x,t)$ is its phase field,
    \item $V_i(\rho_i)$ is a local potential stabilizing $\rho_i$,
    \item $J_{ij} \ge 0$ encodes the strength of interdependence between $i$ and $j$.
\end{itemize}

The action functional is
\begin{equation}
S = \int \mathcal{L}\,\mathrm{d}^d x\,\mathrm{d}t,
\end{equation}
and the dynamics follow from the stationary action principle 
$\delta S = 0$ under suitable variations of $\rho_i$ and $\theta_i$.

\section{Global Phase Symmetry}

\subsection{Definition of the symmetry}

Consider a global shift of all phases:
\begin{equation}
\theta_i(x,t) \;\mapsto\; \theta_i(x,t) + \alpha,
\qquad \forall i,\ \forall x,\ \forall t,
\label{eq:GlobalPhaseShift}
\end{equation}
where $\alpha$ is a constant independent of space, time, and subsystem index.

Under this transformation:
\begin{itemize}
    \item The phase differences $\theta_i - \theta_j$ remain unchanged.
    \item The gradients $\nabla\theta_i$ are unaffected.
    \item The densities $\rho_i$ are unchanged.
\end{itemize}

Since the Lagrangian \eqref{eq:LagrangianEconomic} depends on $\theta_i$ 
only through $\partial_t\theta_i$, $\nabla\theta_i$, and differences 
$\theta_i - \theta_j$, a uniform shift~\eqref{eq:GlobalPhaseShift} leaves 
$\mathcal{L}$ invariant. 
Thus, the economic field exhibits a global phase symmetry.

\subsection{Economic interpretation}

Global phase symmetry reflects the fact that only \emph{relative} orientation 
matters. 
If all subsystems rotate their strategic or expectation state by the same 
constant offset, the economic configuration is unchanged. 
There is no absolute reference direction in phase space; only differences 
drive reallocation and tension.

This expresses an economic principle:
\begin{quote}
\centering
\textit{Value flows and tensions depend on relative misalignment, not on any absolute orientation.}
\end{quote}

\subsection{Associated conservation law: total density}

As in standard field theory, continuous symmetries of the action correspond 
to conserved quantities. 
For global phase symmetry, the conserved quantity is the total density.

To see this, consider the continuity equation obtained by varying $\theta_i$:
\begin{equation}
\frac{\partial \rho_i}{\partial t}
+ \nabla \cdot \left( \rho_i \nabla \theta_i \right)
= \sum_{j \ne i} J_{ij}\,\rho_i \rho_j\,\sin(\theta_i - \theta_j),
\label{eq:rhoEvolutionA2}
\end{equation}
and sum over all $i$:
\begin{equation}
\frac{\partial}{\partial t}
\left( \sum_i \rho_i \right)
+ \nabla \cdot \left( \sum_i \rho_i \nabla \theta_i \right)
= \sum_i \sum_{j \ne i} J_{ij}\,\rho_i \rho_j\,\sin(\theta_i - \theta_j).
\end{equation}
The double sum on the right-hand side cancels due to antisymmetry:
\[
J_{ij}\,\rho_i \rho_j\,\sin(\theta_i - \theta_j)
= -J_{ji}\,\rho_j \rho_i\,\sin(\theta_j - \theta_i),
\]
and $J_{ij} = J_{ji}$.

Therefore,
\begin{equation}
\frac{\partial}{\partial t}\left( \sum_i \rho_i \right)
+ \nabla \cdot \left( \sum_i \rho_i \nabla \theta_i \right)
= 0.
\label{eq:TotalDensityConservation}
\end{equation}

Integrating over the whole domain and assuming appropriate boundary conditions 
(e.g.\ vanishing flux at infinity or periodic boundaries), we obtain:
\begin{equation}
\frac{\mathrm{d}}{\mathrm{d}t}
\int \sum_i \rho_i(x,t)\,\mathrm{d}^d x
= 0.
\end{equation}
Thus, the total density is conserved.

Economically, this means that the field dynamics \emph{redistribute} value but 
do not create or annihilate it internally. 
The global amount of economic presence is an invariant of the motion.

\section{Spatial Translation Symmetry}

\subsection{Homogeneity of the underlying domain}

Consider a spatial translation:
\begin{equation}
x \;\mapsto\; x + a,
\qquad a \in \mathbb{R}^d,
\label{eq:SpatialTranslation}
\end{equation}
with time unchanged. 
If the Lagrangian density does not depend explicitly on $x$ (i.e.\ there are 
no externally imposed spatial inhomogeneities), then 
$\mathcal{L}$ is invariant under~\eqref{eq:SpatialTranslation}.

This reflects an assumption of \emph{homogeneity}: the local rules governing 
interaction and evolution are the same across the domain. 
Spatial structure may still emerge through the configuration of $\rho_i$ and 
$\theta_i$, but it is not hard-coded into $\mathcal{L}$.

\subsection{Momentum-like conserved quantity}

Spatial translation symmetry implies the conservation of a momentum-like 
quantity. 
Formally, one can define a stress-energy tensor $T^{\mu\nu}$ associated with 
the Lagrangian, whose spatial components yield conserved currents.

In the purely economic reading, we restrict attention to the continuity of 
economic flows:
\begin{equation}
\partial_t \rho_i + \nabla \cdot j_i = \text{(interaction terms)},
\end{equation}
with
\[
j_i = \rho_i \nabla\theta_i,
\]
the flux of density of subsystem $i$.

Translation invariance implies that the \emph{integrated} momentum of the field,
constructed from $\rho_i$ and $\nabla\theta_i$, is conserved in the absence of 
external forces or explicit spatial dependence in $V_i$ and $J_{ij}$.

Intuitively, the system cannot generate a net drift in one direction without 
breaking the underlying homogeneity. 
Patterns may propagate and clusters may move, but the aggregate momentum-like 
quantity remains invariant.

\subsection{Economic interpretation}

Spatial translation symmetry captures the idea that the dynamics do not 
privilege any specific location. 
Regions may differ in realized density and phase, but the underlying rules are 
uniform.

Economically, this is a natural abstraction for modeling large-scale structures 
where local differences emerge endogenously, rather than being imposed by 
external spatial biases.

\section{Time Translation Symmetry}

\subsection{Autonomous dynamics}

A key property of the Lagrangian~\eqref{eq:LagrangianEconomic} is that it has 
no explicit dependence on time $t$: it depends on $t$ only through the fields 
$\rho_i(x,t)$ and $\theta_i(x,t)$ and their derivatives.

A time translation:
\begin{equation}
t \;\mapsto\; t + \tau,
\label{eq:TimeTranslation}
\end{equation}
with $\tau$ constant, leaves the action invariant.

This is the statement that the system is \emph{autonomous}: the rules of 
evolution are the same at all times. 
Dynamics may lead to different configurations, but the underlying law does not 
change.

\subsection{Conservation of the Hamiltonian functional}

Time translation symmetry implies the conservation of a Hamiltonian functional 
$H$, often interpreted as the total energy of the field.

For the economic field, $H$ takes the schematic form:
\begin{equation}
H = \int \left[
 \sum_i \frac{1}{2}\,\rho_i\,|\nabla \theta_i|^2
 + \sum_i V_i(\rho_i)
 + \sum_{i < j} J_{ij}\,\rho_i \rho_j\,
   \bigl[1 - \cos(\theta_i - \theta_j)\bigr]
\right] \mathrm{d}^d x.
\label{eq:EconomicHamiltonian}
\end{equation}

This Hamiltonian collects three contributions:
\begin{itemize}
    \item a kinetic-like term associated with phase gradients,
    \item local structural energy from the potentials $V_i$,
    \item interaction energy from phase-dependent couplings.
\end{itemize}

Time translation symmetry then yields:
\begin{equation}
\frac{\mathrm{d}H}{\mathrm{d}t} = 0,
\end{equation}
under the field equations derived from $\mathcal{L}$.

\subsection{Economic meaning of $H$}

The Hamiltonian $H$ measures the global tension and structure of the economic 
field: 
\begin{itemize}
    \item large gradients of phase correspond to high directional tension,
    \item deviations of $\rho_i$ from their structural baselines contribute 
          local strain,
    \item misaligned phases across strongly coupled subsystems raise the 
          interaction energy.
\end{itemize}

Conservation of $H$ in the autonomous model means that the system can 
redistribute and reorganize tension, but not eliminate it entirely through 
internal dynamics alone. 
Relaxation toward coherent states redistributes energy between kinetic, 
potential, and interaction contributions, but the total remains fixed.

\section{Summary of Symmetry Structure}

We summarize the symmetry structure and associated invariants:

\begin{itemize}
    \item \textbf{Global phase symmetry}: 
    invariance under $\theta_i \mapsto \theta_i + \alpha$ for all $i$,
    leading to conservation of total density 
    $\int \sum_i \rho_i\,\mathrm{d}^d x$.

    \item \textbf{Spatial translation symmetry}: 
    invariance under $x \mapsto x + a$, implying the existence of a 
    momentum-like conserved quantity associated with the uniformity 
    of the underlying domain.

    \item \textbf{Time translation symmetry}: 
    invariance under $t \mapsto t + \tau$, yielding conservation of the 
    Hamiltonian functional $H$ in~\eqref{eq:EconomicHamiltonian}.
\end{itemize}

These symmetries define the geometric backbone of the economic field. 
They constrain the possible dynamics, explain the persistence of certain 
macroscopic quantities, and identify coherent states as low-energy 
configurations consistent with the invariants of motion.

Future chapters will build on this structure to analyze coherence clusters, 
oscillatory regimes, and the response of the field to external perturbations.

\end{document}
