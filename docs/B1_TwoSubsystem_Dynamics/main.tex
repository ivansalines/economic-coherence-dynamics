\documentclass[12pt,a4paper]{article}

\usepackage[utf8]{inputenc}
\usepackage[T1]{fontenc}
\usepackage{amsmath,amssymb,amsfonts}
\usepackage{bm}
\usepackage{geometry}
\usepackage{graphicx}
\usepackage{hyperref}

\geometry{margin=2.4cm}

\title{\textbf{Two-Subsystem Economic Dynamics:\\
Coherence, Cycles, and Relaxation}}

\author{Ivan Salines --- Independent Researcher}
\date{November 2025}

\begin{document}
\maketitle

\begin{abstract}
This chapter develops the two-subsystem economic field model as a minimal 
laboratory for coherence, exchange cycles, and relaxation toward aligned 
configurations. Starting from the density--phase Lagrangian, we derive the 
reduced dynamical system for two interacting subsystems, analyze its conserved 
quantities, identify fixed points and their stability, and discuss the 
conditions under which persistent oscillations and phase-locked regimes emerge. 
The model serves both as a conceptual bridge between the field-theoretic 
formulation and numerical experiments, and as a template for understanding 
larger networks of economic subsystems.
\end{abstract}

\tableofcontents

\section{Introduction}

The full economic field is a many-component system in which densities and 
phases interact across a network of couplings. 
To gain intuition about its dynamical behavior, it is useful to consider the 
simplest non-trivial case: two interacting subsystems, labeled $A$ and $B$.

In this reduced setting, the density--phase dynamics can be written as a 
four-dimensional ordinary differential equation (ODE) for 
$\rho_A(t)$, $\rho_B(t)$, $\theta_A(t)$, and $\theta_B(t)$. 
Despite its simplicity, this model already exhibits:
\begin{itemize}
    \item conservation of total density,
    \item relaxation of phase differences,
    \item transient exchange cycles,
    \item the emergence of coherent aligned states.
\end{itemize}

The two-subsystem model thus plays the role of a \emph{dynamical microscope} 
for the economic field, revealing core mechanisms that persist in more 
complex networks.

\section{Two-Subsystem Lagrangian and Equations of Motion}

We consider spatially uniform fields, so that all spatial gradients vanish and 
only temporal evolution remains. 
The Lagrangian for two subsystems $A$ and $B$ reduces to
\begin{equation}
L =
 \rho_A \dot{\theta}_A + \rho_B \dot{\theta}_B
 - V_A(\rho_A) - V_B(\rho_B)
 - J\,\rho_A \rho_B \bigl[1 - \cos(\theta_A - \theta_B)\bigr],
\label{eq:LagrangianTwo}
\end{equation}
where:
\begin{itemize}
    \item $\rho_A(t)$, $\rho_B(t)$ are the densities of the two subsystems,
    \item $\theta_A(t)$, $\theta_B(t)$ are the phases,
    \item $V_A$ and $V_B$ are local potentials,
    \item $J \ge 0$ is the coupling strength.
\end{itemize}

Variation with respect to $\theta_A$ and $\theta_B$ yields:
\begin{align}
\dot{\rho}_A &= - J\,\rho_A \rho_B \,\sin(\theta_A - \theta_B),
\label{eq:rhoAeq}\\
\dot{\rho}_B &= + J\,\rho_A \rho_B \,\sin(\theta_A - \theta_B).
\label{eq:rhoBeq}
\end{align}
Variation with respect to $\rho_A$ and $\rho_B$ gives:
\begin{align}
\dot{\theta}_A &= V_A'(\rho_A)
                 + J \rho_B \bigl[1 - \cos(\theta_A - \theta_B)\bigr],
\label{eq:thetaAeq}\\
\dot{\theta}_B &= V_B'(\rho_B)
                 + J \rho_A \bigl[1 - \cos(\theta_A - \theta_B)\bigr].
\label{eq:thetaBeq}
\end{align}

These four equations form the basic dynamical system.

\section{Conserved Total Density}

Adding Eqs.~\eqref{eq:rhoAeq}--\eqref{eq:rhoBeq} we obtain:
\begin{equation}
\dot{\rho}_A + \dot{\rho}_B
= - J \rho_A \rho_B \sin(\theta_A - \theta_B)
  + J \rho_A \rho_B \sin(\theta_A - \theta_B)
= 0.
\end{equation}

Hence
\begin{equation}
\rho_{\mathrm{tot}} = \rho_A + \rho_B
\end{equation}
is conserved:
\begin{equation}
\frac{\mathrm{d}\rho_{\mathrm{tot}}}{\mathrm{d}t} = 0.
\end{equation}

Economically, the two-subsystem dynamics reshuffle density between $A$ and $B$, 
but do not create or annihilate it. 
The model describes \emph{redistribution} under alignment forces, not 
exogenous growth or depletion.

\section{Phase Difference and Interaction Energy}

It is convenient to introduce the phase difference
\begin{equation}
\Delta\theta = \theta_A - \theta_B.
\end{equation}

The interaction energy for the two subsystems is
\begin{equation}
E_{\mathrm{int}}
= J \rho_A \rho_B \bigl[1 - \cos(\Delta\theta)\bigr].
\end{equation}
This term is minimized when
\begin{equation}
\Delta\theta = 2\pi k,\qquad k \in \mathbb{Z},
\end{equation}
corresponding to perfectly aligned phases.

Differentiating $\Delta\theta$ with respect to time:
\begin{align}
\dot{\Delta\theta}
&= \dot{\theta}_A - \dot{\theta}_B \nonumber\\
&= V_A'(\rho_A) - V_B'(\rho_B)
 + J \bigl[\rho_B - \rho_A\bigr] \bigl[1 - \cos(\Delta\theta)\bigr].
\label{eq:deltathetaeq}
\end{align}

This shows that both the asymmetry in local potentials and differences in 
density contribute to the evolution of the phase difference.

\section{Local Potentials and Structural Baselines}

To obtain a more explicit form, we consider quadratic potentials:
\begin{align}
V_A(\rho_A) &= \frac{1}{2} k_A (\rho_A - \rho_{0A})^2,\\
V_B(\rho_B) &= \frac{1}{2} k_B (\rho_B - \rho_{0B})^2,
\end{align}
so that
\begin{align}
V_A'(\rho_A) &= k_A (\rho_A - \rho_{0A}),\\
V_B'(\rho_B) &= k_B (\rho_B - \rho_{0B}).
\end{align}

These forms stabilize $\rho_A$ and $\rho_B$ around structural baselines 
$\rho_{0A}$ and $\rho_{0B}$, with stiffness controlled by $k_A$ and $k_B$.

Substituting into~\eqref{eq:deltathetaeq} yields:
\begin{equation}
\dot{\Delta\theta}
= k_A (\rho_A - \rho_{0A}) - k_B (\rho_B - \rho_{0B})
 + J (\rho_B - \rho_A)\,[1 - \cos(\Delta\theta)].
\label{eq:deltathetaQuadratic}
\end{equation}

\section{Fixed Points and Coherent Alignment}

Fixed points of the dynamics satisfy:
\begin{align}
\dot{\rho}_A &= 0, & \dot{\rho}_B &= 0,
\label{eq:rhoFixedA}\\
\dot{\theta}_A &= 0, & \dot{\theta}_B &= 0.
\label{eq:thetaFixedA}
\end{align}

From~\eqref{eq:rhoAeq}--\eqref{eq:rhoBeq}, either:
\begin{itemize}
    \item $\rho_A = 0$ or $\rho_B = 0$, or
    \item $\sin(\Delta\theta) = 0$.
\end{itemize}

We are interested in non-degenerate configurations with both densities 
strictly positive, $\rho_A > 0$ and $\rho_B > 0$. 
Then
\begin{equation}
\sin(\Delta\theta^\ast) = 0
\quad\Rightarrow\quad
\Delta\theta^\ast = n \pi,\qquad n \in \mathbb{Z}.
\end{equation}

For the interaction energy,
\begin{equation}
E_{\mathrm{int}}^\ast
= J \rho_A \rho_B \bigl[1 - \cos(\Delta\theta^\ast)\bigr],
\end{equation}
this yields:
\begin{itemize}
    \item if $\Delta\theta^\ast = 2\pi k$, then $E_{\mathrm{int}}^\ast = 0$ 
          (aligned phases, minimal interaction energy),
    \item if $\Delta\theta^\ast = (2k+1)\pi$, then 
          $E_{\mathrm{int}}^\ast = 2J \rho_A \rho_B$ (anti-aligned, maximal 
          interaction energy).
\end{itemize}

Thus, the \emph{coherent fixed points} are those with:
\begin{equation}
\Delta\theta^\ast = 2\pi k, \qquad k\in\mathbb{Z},
\end{equation}
and densities satisfying the stationary conditions from
Eqs.~\eqref{eq:thetaAeq}--\eqref{eq:thetaBeq}.

\section{Linear Stability of the Coherent Fixed Point}

We focus on a coherent fixed point with 
$\Delta\theta^\ast = 0$ and $(\rho_A^\ast,\rho_B^\ast)$ such that:
\begin{align}
0 &= V_A'(\rho_A^\ast),\\
0 &= V_B'(\rho_B^\ast).
\end{align}
For quadratic potentials, this implies 
$\rho_A^\ast = \rho_{0A}$ and $\rho_B^\ast = \rho_{0B}$.

We introduce small perturbations:
\begin{align}
\rho_A(t) &= \rho_A^\ast + \delta\rho_A(t),\\
\rho_B(t) &= \rho_B^\ast + \delta\rho_B(t),\\
\Delta\theta(t) &= 0 + \delta\theta(t),
\end{align}
with $|\delta\rho_A|, |\delta\rho_B|, |\delta\theta| \ll 1$.

Expanding Eqs.~\eqref{eq:rhoAeq}, \eqref{eq:rhoBeq}, 
and~\eqref{eq:deltathetaQuadratic} to linear order in the perturbations, we 
use:
\begin{align}
\sin(\delta\theta) &\approx \delta\theta,\\
1 - \cos(\delta\theta) &\approx \tfrac{1}{2} (\delta\theta)^2 \approx 0 
\quad\text{at leading order}.
\end{align}

Thus, to first order,
\begin{align}
\dot{\delta\rho}_A &\approx - J \rho_A^\ast \rho_B^\ast\, \delta\theta,
\label{eq:lin_rhoA}\\
\dot{\delta\rho}_B &\approx + J \rho_A^\ast \rho_B^\ast\, \delta\theta,
\label{eq:lin_rhoB}
\end{align}
and, since $V_A'(\rho_A^\ast) = V_B'(\rho_B^\ast) = 0$, we have:
\begin{align}
\dot{\delta\theta}
&\approx k_A \delta\rho_A - k_B \delta\rho_B.
\label{eq:lin_dtheta}
\end{align}

It is convenient to define:
\begin{align}
\delta\rho_{\mathrm{tot}} &= \delta\rho_A + \delta\rho_B,\\
\delta\rho_{\mathrm{rel}} &= \delta\rho_A - \delta\rho_B.
\end{align}
From Eqs.~\eqref{eq:lin_rhoA}--\eqref{eq:lin_rhoB}:
\begin{align}
\dot{\delta\rho}_{\mathrm{tot}} &= 0,\\
\dot{\delta\rho}_{\mathrm{rel}} &= -2 J \rho_A^\ast \rho_B^\ast\, \delta\theta.
\end{align}
The total density perturbation is thus conserved at linear order.

Using~\eqref{eq:lin_dtheta} and 
$\delta\rho_A = (\delta\rho_{\mathrm{tot}} + \delta\rho_{\mathrm{rel}})/2$, 
$\delta\rho_B = (\delta\rho_{\mathrm{tot}} - \delta\rho_{\mathrm{rel}})/2$, we obtain:
\begin{equation}
\dot{\delta\theta}
= \frac{1}{2}(k_A - k_B)\,\delta\rho_{\mathrm{tot}}
 + \frac{1}{2}(k_A + k_B)\,\delta\rho_{\mathrm{rel}}.
\end{equation}

If we focus on perturbations that preserve total density at the linear level 
(i.e.\ $\delta\rho_{\mathrm{tot}}=0$), the coupled system reduces to:
\begin{align}
\dot{\delta\rho}_{\mathrm{rel}} &= -2 J \rho_A^\ast \rho_B^\ast\, \delta\theta,
\label{eq:sys_rel_1}\\
\dot{\delta\theta} &= \frac{1}{2}(k_A + k_B)\,\delta\rho_{\mathrm{rel}}.
\label{eq:sys_rel_2}
\end{align}

Differentiating~\eqref{eq:sys_rel_2} with respect to time and using 
\eqref{eq:sys_rel_1} gives:
\begin{equation}
\ddot{\delta\theta}
= \frac{1}{2}(k_A + k_B)\,\dot{\delta\rho}_{\mathrm{rel}}
= - (k_A + k_B)\,J \rho_A^\ast \rho_B^\ast\, \delta\theta.
\end{equation}

Therefore,
\begin{equation}
\ddot{\delta\theta} + \omega^2 \delta\theta = 0,
\qquad \omega^2 = (k_A + k_B)\,J \rho_A^\ast \rho_B^\ast > 0.
\end{equation}

This is a harmonic oscillator equation: small perturbations of the phase 
difference around the coherent fixed point oscillate with frequency $\omega$. 
In the presence of additional damping mechanisms (not included here), these 
oscillations would decay, leading to exponential relaxation toward 
$\Delta\theta = 0$.

\subsection{Qualitative picture}

The coherent fixed point with aligned phases and densities at their 
structural baselines is linearly stable: small deviations in phase and 
relative density lead to bounded oscillations. 
In realistic settings, frictional or adaptation effects would damp these 
oscillations, making the coherent state an attractor.

Economically, this corresponds to two strongly interdependent subsystems 
that settle into a stable, mutually aligned configuration, with exchange 
flows organizing into recurrent and predictable patterns.

\section{Exchange Cycles and Oscillatory Regimes}

The linear analysis shows that small perturbations behave like a harmonic 
oscillator. 
Beyond the linear regime, the full nonlinear dynamics can generate:
\begin{itemize}
    \item larger-amplitude oscillations of $\Delta\theta(t)$,
    \item periodic or quasi-periodic cycles in $\rho_A(t)$ and $\rho_B(t)$,
    \item transitions between different phase-locked regimes.
\end{itemize}

These oscillatory regimes can be interpreted as \emph{exchange cycles}: 
the two subsystems repeatedly transfer density back and forth, driven by 
phase misalignment and then pulled back toward coherence by the alignment 
force.

Depending on the shape of the potentials $V_A$ and $V_B$, and on the coupling 
strength $J$, the system may exhibit:
\begin{itemize}
    \item fast or slow relaxation to coherence,
    \item persistent oscillations around the coherent state,
    \item metastable patterns in which one subsystem temporarily dominates.
\end{itemize}

\section{Connection to Numerical Experiments}

The two-subsystem model can be implemented numerically using standard ODE 
integrators. 
For example, one may choose:
\begin{itemize}
    \item quadratic potentials with given $(k_A,k_B,\rho_{0A},\rho_{0B})$,
    \item initial densities $\rho_A(0)$, $\rho_B(0)$ with 
          $\rho_A(0) + \rho_B(0) = \rho_{\mathrm{tot}}$,
    \item a non-zero initial phase difference $\Delta\theta(0)$.
\end{itemize}

The resulting trajectories show:
\begin{itemize}
    \item that $\rho_A(t) + \rho_B(t)$ remains approximately constant,
    \item that $\Delta\theta(t)$ tends to fluctuate around a coherent value,
    \item how the characteristic frequency of oscillations depends on 
          $J$, $k_A$, $k_B$, and the structural baselines.
\end{itemize}

Such numerical experiments provide a concrete, visual representation of the 
abstract analysis carried out in this chapter, and form a bridge between 
the analytic theory and applications to more complex multi-agent networks.

\section{Conclusions}

The two-subsystem economic model encapsulates, in minimal form, the essence 
of density--phase dynamics:

\begin{itemize}
    \item conservation of total density as a consequence of phase symmetry,
    \item coherent fixed points with aligned phases and structural densities,
    \item oscillatory regimes around these coherent states,
    \item exchange cycles driven by misalignment and constrained by coupling.
\end{itemize}

This model serves as a foundational building block for the study of larger 
economic networks, where many subsystems interact. 
The mechanisms of coherence, oscillation, and relaxation observed here will 
reappear in richer forms when we consider clusters, modular structures, and 
global economic fields.

\end{document}
