\documentclass[12pt,a4paper]{article}

\usepackage[utf8]{inputenc}
\usepackage[T1]{fontenc}
\usepackage{amsmath,amssymb,amsfonts}
\usepackage{bm}
\usepackage{geometry}
\usepackage{graphicx}
\usepackage{hyperref}
\usepackage{tikz}
\usetikzlibrary{calc,positioning}
\usepackage{pgfplots}
\pgfplotsset{compat=1.18}

\geometry{margin=2.4cm}

\title{\textbf{Foundations of the Economic Field:\\
Density, Phase, and the Geometry of Coherent Exchange}}

\author{Ivan Salines --- Independent Researcher}
\date{November 2025}

\begin{document}
\maketitle

\begin{abstract}
This document introduces the foundational structure of the \emph{economic field}, 
a continuum formulation in which economic subsystems are represented by coupled scalar 
fields: a density field and a phase field. 
The purpose of this first chapter is to define these fields, establish their 
interpretation, and derive the geometric and dynamical principles that govern 
their interactions. 
The construction is parallel in spirit to the field-theoretic formulation of 
organized motion developed in physics, but remains fully autonomous and 
distilled to the essential phenomenology of economic coordination. 
Coherence, alignment, and circulation arise not as imposed structures but as 
energetically favorable configurations of the field.
\end{abstract}

\tableofcontents

\section{Introduction}

Economic systems are commonly described in terms of aggregated variables, 
representative agents, or discrete multi-sector frameworks. 
Yet real economic organization displays large-scale coherence: 
long-lived exchange patterns, persistent directional flows, cluster formation, 
and organized modes of interdependence.

Such coherence suggests the presence of an underlying geometric structure---a 
field-theoretic substrate in which macroscopic order emerges from local 
interactions.

In this document we introduce the \emph{economic field}, composed of:
\begin{itemize}
    \item a \textbf{density field} $\rho(x,t)$,
    \item a \textbf{phase field} $\theta(x,t)$,
\end{itemize}
defined on a continuum domain, which together form the minimal description of 
organized economic motion.  
These fields are not metaphors: they are dynamical variables whose interactions 
give rise to observable macroeconomic patterns.

\section{The Density Field \texorpdfstring{$\rho(x,t)$}{rho(x,t)}}

The density field $\rho(x,t)$ represents the distribution of economic resources, 
capacity, or potential at point $x$ and time $t$.  
While its interpretation can be adapted to context, its role is invariant:

\begin{quote}
\centering
\textit{Density measures the intensity of local economic presence.}
\end{quote}

High density corresponds to a region or subsystem with substantial productive 
capability, structural significance, or accumulated potential.  
Low density indicates scarcity, fragility, or underdeveloped infrastructure.

Mathematically, $\rho(x,t)$ behaves like a conserved scalar field: in the absence 
of external injection, its integral over the whole domain remains constant. 
This reflects the principle that economic value is internally redistributed, not 
created or destroyed by the dynamics itself.

\subsection{Local Potentials}

Each subsystem is endowed with a local potential $V(\rho)$ which captures 
structural or technological constraints. 
Typical choices stabilize $\rho$ near an operational reference value $\rho_0$.

A minimal example is the quadratic potential:
\[
V(\rho) = \frac{1}{2} k (\rho - \rho_0)^2,
\]
where $k$ measures structural rigidity.

\section{The Phase Field \texorpdfstring{$\theta(x,t)$}{theta(x,t)}}

The phase field encodes orientation, expectations, or strategic direction.
Its interpretation is subtle but essential:

\begin{quote}
\centering
\textit{Phase is the directional state of the subsystem, governing how density flows.}
\end{quote}

Changes in $\theta$ correspond to modifications in strategy, alignment, or local 
economic directionality.

Gradients of phase,
\[
\nabla\theta(x,t),
\]
drive flows of density, representing directional reallocation or shifts in 
activity.

The phase field is not arbitrary: it is dynamically driven by internal potential 
forces and by phase differences between interdependent subsystems.

\section{The Economic Lagrangian}

The economic field is governed by a Lagrangian density of the form:
\begin{equation}
\mathcal{L}
= \rho\,\frac{\partial\theta}{\partial t}
 - \frac{1}{2}\,\rho\,|\nabla\theta|^2
 - V(\rho)
 - \sum_{j} J_{j}\,\rho\,\rho_j\,[1 - \cos(\theta - \theta_j)].
\end{equation}

This structure encodes:
\begin{itemize}
    \item temporal evolution of phase weighted by density,
    \item energetic cost for spatial misalignment,
    \item structural constraints via the potential $V(\rho)$,
    \item pairwise alignment forces via the cosine coupling.
\end{itemize}

The geometric meaning is profound: coherence is energetically favored; 
fragmentation is costly.

\section{Euler--Lagrange Equations}

Variation with respect to $\theta$ yields a continuity equation:
\[
\partial_t \rho + \nabla \cdot (\rho \nabla\theta)
= \sum_j J_j\,\rho\,\rho_j\,\sin(\theta - \theta_j).
\]

Variation with respect to $\rho$ yields a Hamilton--Jacobi-type equation:
\[
\partial_t \theta
+ \frac{1}{2}|\nabla\theta|^2
+ V'(\rho)
+ \sum_j J_j\,\rho_j\,[1 - \cos(\theta - \theta_j)]
= 0.
\]

Together these describe the full dynamics of the economic field.

\section{Interpretation and Geometry of Coherence}

\subsection{Alignment Forces}

The interaction term promotes alignment of phases.  
At equilibrium, interdependent subsystems settle into coherent orientations:
\[
\theta \approx \theta_j.
\]

Misalignment generates tension and induces reallocation flows.

\subsection{Coherence Clusters}

Strongly coupled subsystems form clusters in which phases lock together.  
These clusters behave as macro-entities, maintaining identity and resisting 
perturbations.

\subsection{Efficient Exchange Cycles}

When phase differences vanish, density flows become cyclic and recurrent, 
producing stable exchange structures.

\section{Summary}

The economic field is the foundational geometric structure from which large-scale 
economic order emerges.  
Density and phase jointly encode resource distribution and strategic orientation, 
while the Lagrangian formalism endows the system with endogenous dynamics.

This first chapter establishes the core conceptual and mathematical substrate on 
which the remainder of the economic coherence theory is built.
Future chapters will explore interaction networks, cluster dynamics, oscillatory 
regimes, and the geometry of coherent macroeconomic organization.

\end{document}
